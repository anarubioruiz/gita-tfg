\chapter{Resumen}

El presente documento es un ejemplo de memoria del Trabajo de Fin Grado según el formato y criterios del Grado en Ingeniería Informática de la Facultad de Ciencias Sociales y Tecnologías de la Información de Talavera de la Reina. La intención es que este texto sirva además como una serie de consejos sobre tipografía, \LaTeX, redacción y estructura de la memoria que podrían resultar de ayuda. Por este motivo, se aconseja al lector consultar también el código fuente de este documento, que usa la clase \LaTeX{} \gitatfg{}:

 \url{https://github.com/anarubioruiz/gita-tfg}

Si encuentras cualquier error o tienes alguna sugerencia, por favor, utiliza el \emph{issue tracker} del proyecto \gitatfg{} en:

\url{https://github.com/anarubioruiz/gita-tfg/issues}

El resumen debería estar formado por dos o tres párrafos resaltando lo más destacable del documento. No es una introducción al problema, es decir, debería incluir los logros más importantes del proyecto. Suele ser más sencillo escribirlo cuando la memoria está prácticamente terminada. Debería caber en esta página (es decir, esta cara).


\chapter{Abstract}

English version of the previous page.

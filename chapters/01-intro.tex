\chapter{Introducción}

\drop{E}{sto} se llama «letra capital» y debería utilizarse únicamente al comienzo de capítulo como artificio decorativo. Para que resulte estéticamente adecuada, este primer párrafo debería tener más del doble de líneas de lo que ocupe verticalmente la letra capital (dos en este caso).

El capítulo de introducción debe dar al lector una perspectiva básica \textemdash pero completa\textemdash~del problema que se pretende abordar, de la motivación, y también de la estrategia y enfoque que el autor propone para su resolución. El lector debería poder determinar si este documento le interesa leyendo únicamente este capítulo.

\section{Motivación}

De acuerdo con la normativa, en este apartado se debe:

\begin{quote}
  «\dots señalar la necesidad de la que surge el trabajo, su actualidad y pertinencia. Puede incluir un estado de la cuestión en la que se aborden estudios o desarrollos previos y en qué medida sirven de base al trabajo que se presenta».
\end{quote}

Por cierto, las comillas que deben usarse en castellano son las «latinas»\footnote{Las comillas latinas se pueden escribir pulsado AltGr-z y AltGr-x en la gran mayoría de editores.}, dejando las ``inglesas'' para los raros casos en los que aparezca una cita en el cuerpo de otra~\cite{sousa}. Además, en la portada \textemdash y otras páginas de presentación\textemdash~el nombre o título del proyecto debe aparecer sin comillas, cursiva u otros formatos. Si se cita el título de otra obra o el nombre de un capítulo, sí debe aparecer entre comillas.

\section{Objetivos}

\noindent  % <-- Elimina la sangría de la primera línea
En este capítulo la normativa indica que \emph{«para hacer un planteamiento apropiado de los objetivos se recomienda utilizar la guía para la elaboración de propuestas de \acf{TFG} en la que se explica cómo definir correctamente los objetivos de un \ac{TFG}»}. Puedes consultar la guía en la página web de la Facultad\footnote{\url{https://www.uclm.es/toledo/fcsociales/grado-informatica/trabajo-fin-de-grado}}.


\section{Estructura del documento}

Pueden incluirse aquí una sección con algunos consejos para la lectura del documento dependiendo de la motivación o conocimientos del lector.  También puede ser útil incluir una lista con el nombre y finalidad de cada uno de los capítulos restantes. Una posible lista de capítulos sería la siguiente:

\begin{definitionlist}
\item[Capítulo \ref{chap:antecedentes}: \nameref{chap:antecedentes}] Revisión de la literatura y estado del arte.
\item[Capítulo \ref{chap:metodologia}: \nameref{chap:metodologia}] Descripción de la metodología empleada.
\item[Capítulo \ref{chap:resultados}: \nameref{chap:resultados}] Presentación de los resultados obtenidos.
\item[Capítulo \ref{chap:conclusiones}: \nameref{chap:conclusiones}] Conclusiones y trabajo futuro.
\end{definitionlist}


% Local Variables:
%  coding: utf-8
%  mode: latex
%  mode: flyspell
%  ispell-local-dictionary: "castellano8"
% End:
